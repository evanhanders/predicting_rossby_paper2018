%\documentclass[iop]{emulateapj}
\documentclass[aps, prl, twocolumn, nofootinbib, groupedaddress, amsfonts, amssymb, amsmath]{revtex4-1}
\usepackage{graphicx}
\usepackage{bm}
\usepackage{natbib}
%\usepackage[colorlinks=True, linkcolor=blue, citecolor=blue]{hyperref}
%\usepackage[all]{hypcap}

\bibliographystyle{apsrev}

\newcommand{\Div}[1]{\ensuremath{\nabla\cdot\left( #1\right)}}
\newcommand{\angles}[1]{\ensuremath{\left\langle #1 \right\rangle}}
\newcommand{\grad}{\ensuremath{\nabla}}
\newcommand{\RB}{Rayleigh-B\'{e}nard }
\newcommand{\stressT}{\ensuremath{\bm{\bar{\bar{\Pi}}}}}
\newcommand{\lilstressT}{\ensuremath{\bm{\bar{\bar{\sigma}}}}}
\newcommand{\nrho}{\ensuremath{n_{\rho}}}
\newcommand{\approptoinn}[2]{\mathrel{\vcenter{
	\offinterlineskip\halign{\hfil$##$\cr
	#1\propto\cr\noalign{\kern2pt}#1\sim\cr\noalign{\kern-2pt}}}}}

\newcommand{\appropto}{\mathpalette\approptoinn\relax}

\newcommand\mnras{{MNRAS}}%
          % Monthly Notices of the RAS

\begin{document}
%%%%% Create nice title and abstract
\author{Evan H. Anders}
\affiliation{Department of Astrophysical \& Planetary Sciences, University of Colorado -- Boulder}
\affiliation{Laboratory for Atmospheric and Space Physics, Boulder, CO}
\author{Benjamin P. Brown}
\affiliation{Department of Astrophysical \& Planetary Sciences, University of Colorado -- Boulder}
\affiliation{Laboratory for Atmospheric and Space Physics, Boulder, CO}
\title{Convective properties of rotating, stratified atmospheres at low and high Mach number}

\begin{abstract}
Abstract goes here.
\end{abstract}
\maketitle


%%%%% Body of the paper
\section{Introduction}
\refstepcounter{section}
\label{sec:intro}
Papers on rotating RB convection that might matter:
\cite{hathaway&somerville1983, julien&all1996, zhong&all2009, julien&all2012, stellmach&all2014}.
Papers on rotating stratified f-planes that might matter:
\cite{brummell&all1996, brummell&all1998 calkins&all2015a}.  Honestly, most stratified, rotating sims
seem to avoid simplicities (like the f-plane) as though they were the plague.

\section{Experiment} 
\refstepcounter{section}
\label{sec:experiment}
We study a plane-parallel atmosphere whose initial stratification is polytropic,
\begin{equation}
\begin{split}
\rho_0(z) &= \rho_{t}(1 + L_z - z)^m, \\
T_0(z)    &= T_{t}(1 + L_z - z),
\label{eqn:polytrope}
\end{split}
\end{equation}

We utilize a localized f-plane model in which the domain
is subject to a constant, unidirectional rotation, 
$\bm{\Omega} = (0, \Omega_0 \sin\theta, \Omega_0\cos\theta)$.
This is similar to the approached used in \cite{brummell&all1996}, and
implies that the domain represents an area of relatively small extent near
the surface of a planet or star, such that the rotation vector does not
change appreciably over the domain.  This assumption means that
the centripetal force, 
$\bm{F}_{\text{cent}} \propto \bm{\Omega}\times(\bm{\Omega}\times\bm{r})$,
is uniform across any given slice of the atmosphere, and thus it only modifies the
initial pressure profile.  If $\theta \neq 0$, then the upwards component of the
centripetal force acts to weaken the effects of gravity.

We evolve the Fully Compressible Navier-Stokes equations,
\begin{align}
&\begin{aligned}
&\frac{\partial \ln\rho}{\partial t} + \grad\cdot\bm{u} 
    = -\bm{u}\cdot\grad\ln\rho,
	\label{eqn:continuity_eqn}
\end{aligned}\\
&\begin{aligned}
\frac{\partial\bm{u}}{\partial t} + 2\bm{\Omega}&\times\bm{u} + \grad T - 
\nu\grad\cdot\lilstressT - \lilstressT\cdot\grad\nu = \\
&-\bm{u}\cdot\grad\bm{u} - T\grad\ln\rho + \bm{g} + 
\nu\lilstressT\cdot\grad\ln\rho,
\label{eqn:momentum_eqn}
\end{aligned}\\
&\begin{aligned}
\frac{\partial T}{\partial t} -\frac{1}{c_V}\left(\right.\chi&\left.
    \grad^2 T + \grad T\cdot\grad\chi\right) = \\
	&-\bm{u}\cdot\grad T - (\gamma-1)T\grad\cdot{\bm{u}} \\
	&+ \frac{1}{c_V}\left(\chi\grad T \cdot\grad\ln\rho +
	\nu\left[\lilstressT\cdot\nabla\right]\cdot\bm{u}\right), 
	\label{eqn:energy_eqn}
\end{aligned}
\end{align}
with the viscous stress tensor given by
\begin{equation}
\sigma_{ij} \equiv \left(\frac{\partial u_i}{\partial x_j} + 
\frac{\partial u_j}{\partial x_i} - \frac{2}{3}\delta_{ij}\grad\cdot\bm{u}\right).
	\label{eqn:stress_tensor}
\end{equation}

We have previously shown that the superadiabaticity of a polytropic atmosphere, $\epsilon$,
is the primary control parameter of the Mach number of the evolved atmosphere.  Here we study
two values of $\epsilon = (10^{-4}, 0.5)$ in order to study low at high Mach number flows.  We
study the behavior of convection near onset and with a supercriticality parameter up to $10^3$.
We examine three different values of the Convective Rossby number
(Ro$_c = (0.1, 1, 10)$) in order to understand the behavior of atmospheres in which convective
driving dominates over rotation, in which those forces are equal, and in which the rotation
dominates over convective driving.

We measure the resulting Rossby number, Nusselt number, and Reynolds number of all flows in order to
understand the various regimes of convection which are open to us.

\section{Results \& Discussion}
\refstepcounter{section}
\label{sec:results}
This is where figures go and other important things that we like to talk about.



\subsection{acknowledgements}
EHA acknowledges the support of the University of Colorado's George 
Ellery Hale Graduate Student Fellowship.
This work was additionally supported by  NASA LWS grant number NNX16AC92G.  
Computations were conducted 
with support by the NASA High End Computing (HEC) Program through the NASA 
Advanced Supercomputing (NAS) Division at Ames Research Center on Pleiades
with allocations GID s1647.
We thank  Jeff Oishi for many useful discussions. 

\bibliography{../biblio.bib}
\end{document}
