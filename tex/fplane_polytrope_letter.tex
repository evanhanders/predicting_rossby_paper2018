%\documentclass[iop]{emulateapj}
\documentclass[aps, prl, twocolumn, nofootinbib, groupedaddress, amsfonts, amssymb, amsmath]{revtex4-1}
\usepackage{graphicx}
\usepackage{bm}
\usepackage{natbib}
%\usepackage[colorlinks=True, linkcolor=blue, citecolor=blue]{hyperref}
%\usepackage[all]{hypcap}

\bibliographystyle{apsrev}

\newcommand{\Div}[1]{\ensuremath{\nabla\cdot\left( #1\right)}}
\newcommand{\angles}[1]{\ensuremath{\left\langle #1 \right\rangle}}
\newcommand{\grad}{\ensuremath{\nabla}}
\newcommand{\RB}{Rayleigh-B\'{e}nard }
\newcommand{\stressT}{\ensuremath{\bm{\bar{\bar{\Pi}}}}}
\newcommand{\lilstressT}{\ensuremath{\bm{\bar{\bar{\sigma}}}}}
\newcommand{\nrho}{\ensuremath{n_{\rho}}}
\newcommand{\approptoinn}[2]{\mathrel{\vcenter{
	\offinterlineskip\halign{\hfil$##$\cr
	#1\propto\cr\noalign{\kern2pt}#1\sim\cr\noalign{\kern-2pt}}}}}

\newcommand{\appropto}{\mathpalette\approptoinn\relax}

\newcommand\mnras{{MNRAS}}%
          % Monthly Notices of the RAS

\begin{document}
%%%%% Create nice title and abstract
\author{Evan H. Anders}
\affiliation{Department of Astrophysical \& Planetary Sciences, University of Colorado -- Boulder}
\affiliation{Laboratory for Atmospheric and Space Physics, Boulder, CO}
\author{Benjamin P. Brown}
\affiliation{Department of Astrophysical \& Planetary Sciences, University of Colorado -- Boulder}
\affiliation{Laboratory for Atmospheric and Space Physics, Boulder, CO}
\title{Convective heat transport in stratified atmospheres at low and high Mach number}

\begin{abstract}
Abstract goes here
\end{abstract}
\maketitle


%%%%% Body of the paper
\section{Introduction}
\refstepcounter{section}
\label{sec:intro}
Blah blah relevant background and other stuff

\section{Experiment} 
\refstepcounter{section}
\label{sec:experiment}
Here's a polytrope:
\begin{equation}
\begin{split}
\rho_0(z) &= \rho_{t}(1 + L_z - z)^m, \\
T_0(z)    &= T_{t}(1 + L_z - z),
\label{eqn:polytrope}
\end{split}
\end{equation}

We evolve the Fully Compressible Navier-Stokes equations,
\begin{align}
&\begin{aligned}
&\frac{\partial \ln\rho}{\partial t} + \grad\cdot\bm{u} 
    = -\bm{u}\cdot\grad\ln\rho,
	\label{eqn:continuity_eqn}
\end{aligned}\\
&\begin{aligned}
\frac{\partial\bm{u}}{\partial t} + \grad T - 
&\nu\grad\cdot\lilstressT - \lilstressT\cdot\grad\nu = \\
&-\bm{u}\cdot\grad\bm{u} - T\grad\ln\rho + \bm{g} + 
\nu\lilstressT\cdot\grad\ln\rho,
\label{eqn:momentum_eqn}
\end{aligned}\\
&\begin{aligned}
\frac{\partial T}{\partial t} -\frac{1}{c_V}\left(\right.\chi&\left.
    \grad^2 T + \grad T\cdot\grad\chi\right) = \\
	&-\bm{u}\cdot\grad T - (\gamma-1)T\grad\cdot{\bm{u}} \\
	&+ \frac{1}{c_V}\left(\chi\grad T \cdot\grad\ln\rho +
	\nu\left[\lilstressT\cdot\nabla\right]\cdot\bm{u}\right), 
	\label{eqn:energy_eqn}
\end{aligned}
\end{align}
with the viscous stress tensor given by
\begin{equation}
\sigma_{ij} \equiv \left(\frac{\partial u_i}{\partial x_j} + 
\frac{\partial u_j}{\partial x_i} - \frac{2}{3}\delta_{ij}\grad\cdot\bm{u}\right).
	\label{eqn:stress_tensor}
\end{equation}

\section{Results \& Discussion}
\refstepcounter{section}
\label{sec:results}
This is where figures go and other important things that we like to talk about.



\subsection{acknowledgements}
EHA acknowledges the support of the University of Colorado's George 
Ellery Hale Graduate Student Fellowship.
This work was additionally supported by  NASA LWS grant number NNX16AC92G.  
Computations were conducted 
with support by the NASA High End Computing (HEC) Program through the NASA 
Advanced Supercomputing (NAS) Division at Ames Research Center on Pleiades
with allocations GID s1647.
We thank  Jeff Oishi for many useful discussions. 

\bibliography{../biblio.bib}
\end{document}
